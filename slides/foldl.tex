\begin{frame}
\frametitle{foldl}
The \lstinline[basicstyle=\ttfamily]$foldl$ function accepts three values:
\begin{enumerate}
\item \lstinline[basicstyle=\ttfamily]$f :: B => A => B$
\item \lstinline[basicstyle=\ttfamily]$z :: B$
\item \lstinline[basicstyle=\ttfamily]$list :: List[A]$
\end{enumerate}
to get back a value of the type \lstinline[basicstyle=\ttfamily]$B$.

\hrulefill

\lstinline[basicstyle=\ttfamily]$foldLeft :: (B => A => B) => B => List[A] => B$
\lstinline[basicstyle=\ttfamily]$B FoldLeft<A, B>(Func<B, A, B>, B, List<A>)$
\end{frame}

\begin{frame}
\frametitle{foldl}
\begin{block}{?}
\begin{center}
How does \lstinline[basicstyle=\ttfamily]$foldl$ take three values to that return value?
\end{center}
\end{block}
\end{frame}

\begin{frame}[fragile]
\frametitle{foldl}
\begin{block}{all left folds are loops}
\begin{lstlisting}[style=haskell,basicstyle=\scriptsize\ttfamily,mathescape]
(f, z, list) =>
  var r = z
  foreach(a in list)
    r = f(r, a)
  return r
\end{lstlisting}
\end{block}
\end{frame}

\begin{frame}[fragile]
\frametitle{foldl}
\begin{block}{all left folds are loops}
\begin{lstlisting}[style=haskell,basicstyle=\scriptsize\ttfamily,mathescape]
(f, z, list) =>
  var r = `z`
  foreach(a in `list`)
    r = `f`(r, a)
  return r
\end{lstlisting}
\end{block}
\end{frame}

\begin{frame}[fragile]
\frametitle{foldl}
\begin{block}{refactor some loops}
\emph{let's look at a real code example}
\end{block}
\end{frame}

\begin{frame}[fragile]
\frametitle{foldl}
\begin{block}{all left folds are loops}
Let's sum the integers of a list
\end{block}
\end{frame}

\begin{frame}[fragile]
\frametitle{foldl}
\begin{block}{sum the integers of a list}
\begin{lstlisting}[style=haskell,basicstyle=\scriptsize\ttfamily,mathescape]
(f, z, list) =>
  var r = `z`
  foreach(a in `list`)
    r = `f`(r, a)
  return r
\end{lstlisting}
\end{block}
\begin{center}
\LARGE
?
\end{center}
\end{frame}

\begin{frame}[fragile]
\frametitle{foldl}
\begin{block}{sum the integers of a list}
\begin{lstlisting}[style=haskell,basicstyle=\scriptsize\ttfamily,mathescape]
list =>
  var r = `0`
  foreach(a in `list`)
    r = `+`(r, a)
  return r
\end{lstlisting}
\end{block}
\begin{center}
\LARGE
\emph{Replace the values in the loop}
\end{center}
\end{frame}

\begin{frame}[fragile]
\frametitle{foldl}
\begin{block}{sum the integers of a list}
\begin{lstlisting}[style=haskell,basicstyle=\scriptsize\ttfamily,mathescape]
sum list = foldl (\r a -> (+) r a) 0 list
sum = foldl (+) 0
\end{lstlisting}
\end{block}
\end{frame}

\begin{frame}[fragile]
\frametitle{foldl}
\begin{block}{multiply the integers of a list}
\begin{lstlisting}[style=haskell,basicstyle=\scriptsize\ttfamily,mathescape]
\f z list ->
  var r = `z`
  foreach(a in `list`)
    r = `f`(r, a)
  return r
\end{lstlisting}
\end{block}
\begin{center}
\LARGE
?
\end{center}
\end{frame}

\begin{frame}[fragile]
\frametitle{foldl}
\begin{block}{multiply the integers of a list}
\begin{lstlisting}[style=haskell,basicstyle=\scriptsize\ttfamily,mathescape]
\list ->
  var r = `1`
  foreach(a in `list`)
    r = `*`(r, a)
  return r
\end{lstlisting}
\end{block}
\begin{center}
\LARGE
\emph{Replace the values in the loop}
\end{center}
\end{frame}

\begin{frame}[fragile]
\frametitle{foldl}
\begin{block}{multiply the integers of a list}
\begin{lstlisting}[style=haskell,basicstyle=\scriptsize\ttfamily,mathescape]
product list = foldl (\r a -> (*) r a) 1 list
product = foldl (*) 1
\end{lstlisting}
\end{block}
\end{frame}

\begin{frame}[fragile]
\frametitle{foldl}
\begin{block}{all left folds are loops}
Let's reverse a list
\end{block}
\end{frame}

\begin{frame}[fragile]
\frametitle{foldl}
\begin{block}{reverse a list}
\begin{lstlisting}[style=haskell,basicstyle=\scriptsize\ttfamily,mathescape]
\f z list ->
  var r = `z`
  foreach(a in `list`)
    r = `f`(r, a)
  return r
\end{lstlisting}
\end{block}
\begin{center}
\LARGE
?
\end{center}
\end{frame}

\begin{frame}[fragile]
\frametitle{foldl}
\begin{block}{reverse a list}
\begin{lstlisting}[style=haskell,basicstyle=\scriptsize\ttfamily,mathescape]
\list ->
  var r = `Nil`
  foreach(a in `list`)
    r = `flipCons`(r, a)
  return r

flipCons = \r a -> Cons a r
\end{lstlisting}
\end{block}
\begin{center}
\LARGE
\emph{Replace the values in the loop}
\end{center}
\end{frame}

\begin{frame}[fragile]
\frametitle{foldl}
\begin{block}{reverse a list}
\begin{lstlisting}[style=haskell,basicstyle=\scriptsize\ttfamily,mathescape]
reverse list = foldl (\r a -> Cons a r) Nil list
reverse = foldl (flip Cons) Nil
\end{lstlisting}
\end{block}
\end{frame}

\begin{frame}[fragile]
\frametitle{foldl}
\begin{block}{all left folds are loops}
Let's compute the length of a list
\end{block}
\end{frame}

\begin{frame}[fragile]
\frametitle{foldl}
\begin{block}{length of a list}
\begin{lstlisting}[style=haskell,basicstyle=\scriptsize\ttfamily,mathescape]
\f z list ->
  var r = `z`
  foreach(a in `list`)
    r = `f`(r, a)
  return r
\end{lstlisting}
\end{block}
\begin{center}
\LARGE
?
\end{center}
\end{frame}

\begin{frame}[fragile]
\frametitle{foldl}
\begin{block}{length of a list}
\begin{lstlisting}[style=haskell,basicstyle=\scriptsize\ttfamily,mathescape]
\list ->
  var r = `0`
  foreach(a in `list`)
    r = `plus1`(r, a)
  return r

plus1 = \r a -> r + 1
\end{lstlisting}
\end{block}
\begin{center}
\LARGE
\emph{Replace the values in the loop}
\end{center}
\end{frame}

\begin{frame}[fragile]
\frametitle{foldl}
\begin{block}{length of a list}
\begin{lstlisting}[style=haskell,basicstyle=\scriptsize\ttfamily,mathescape]
length list = foldl (\r a -> r + 1) 0 list
length = foldl (const . (+1)) 0
\end{lstlisting}
\end{block}
\end{frame}

\begin{frame}
\frametitle{foldl}
\begin{block}{refactoring, intuition}
\begin{itemize}
\item<1-> a left fold is what you would write if I insisted you remove all duplication from your loops
\item<2-> all left folds are exactly this loop
\item<3-> any question we might ask about a left fold, can be asked about this loop
\end{itemize}
\end{block}
\end{frame}

\begin{frame}
\frametitle{foldl}
\begin{block}{some observations}
\begin{itemize}
\item<1-> a left fold will \textbf{never} work on an infinite list
\item<2-> a correct intuition for left folds is easy to build on existing programming knowledge (loop)
\end{itemize}
\end{block}
\end{frame}

\begin{frame}
\frametitle{foldl}
\begin{center}
Folding to the left does a loop
\end{center}
\end{frame}
