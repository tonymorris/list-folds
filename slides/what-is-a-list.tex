\begin{frame}
\frametitle{Lists}
\begin{center}
What, exactly is a list?
\end{center}
\end{frame}

\begin{frame}
\frametitle{Lists}
\begin{block}{a list is either}
\begin{itemize}
\item a \textbf{Nil} construction, with no associated data
\item a \textbf{Cons} construction, associated with one arbitrary value, and another list
\end{itemize}
\end{block}
\emph{And \textbf{never, ever} anything else}
\end{frame}

\begin{frame}
\frametitle{Lists}
\begin{block}{A List that holds elements of type \lstinline{a} is constructed by either:}
  \begin{itemize}
  \item \lstinline{Nil: List[A]}
  \item \lstinline{Cons: A => List[A] => List[A]}
  \end{itemize}
\end{block}
\end{frame}

\begin{frame}[fragile]
\frametitle{Lists}
\begin{block}{a list declaration using Scala}
\begin{lstlisting}[style=scala,basicstyle=\scriptsize\ttfamily,mathescape]
sealed trait List[A]
case class Nil[A]() extends List[A]
case class Cons[A](h: A, t: List[A])
\end{lstlisting}
\end{block}
\end{frame}

\begin{frame}[fragile]
\frametitle{Some examples of Lists}
\begin{block}{Scala}
\begin{lstlisting}[style=scala,basicstyle=\scriptsize\ttfamily,mathescape]
Cons(12, Nil)
\end{lstlisting}
\end{block}
\begin{block}{printed}
\begin{lstlisting}[style=scala,basicstyle=\scriptsize\ttfamily,mathescape]
[12]
\end{lstlisting}
\end{block}
\end{frame}

\begin{frame}[fragile]
\frametitle{Some examples of Lists}
\begin{block}{Scala}
\begin{lstlisting}[style=scala,basicstyle=\scriptsize\ttfamily,mathescape]
Cons('a', Cons('b', Cons('c', Nil)))
\end{lstlisting}
\end{block}
\begin{block}{printed}
\begin{lstlisting}[style=scala,basicstyle=\scriptsize\ttfamily,mathescape]
['a', 'b', 'c']
\end{lstlisting}
\end{block}
\end{frame}

\begin{frame}
\frametitle{Some nomenclature}
\begin{block}{Naming conventions}
\begin{itemize}
\item<1-> Sometimes you will see \lstinline{Nil} denoted \lstinline{[]}
\item<2-> and \lstinline{Cons} denoted \lstinline{:} which is used in infix position
\item<3-> like this \lstinline{1:(2:(3:[]))}
\item<4-> but this is the same data structure
\end{itemize}
\end{block}
\end{frame}
