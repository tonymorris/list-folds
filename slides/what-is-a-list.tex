\begin{frame}
\frametitle{Lists}
\begin{block}{What is a list?}
What, exactly is a list?
\end{block}
\end{frame}

\begin{frame}
\frametitle{Lists}
\begin{block}{a list is either}
\begin{itemize}
\item a \textbf{Nil} construction, with no associated data
\item a \textbf{Cons} construction, associated with one arbitrary value, and another list
\end{itemize}
\end{block}
\emph{And \textbf{never, ever} anything else}
\end{frame}

\begin{frame}[fragile]
\frametitle{Lists}
\begin{block}{a list using C\#}
\begin{lstlisting}[style=csharp,basicstyle=\scriptsize\ttfamily,mathescape]
interface List<A>{}
class Nil<A> : List<A> {}
class Cons<A> : List<A> { A head; List<A> tail; }
\end{lstlisting}
\end{block}
\tiny{\emph{And some tricks to enforce \textbf{never ever anything else}}}
\end{frame}

\begin{frame}[fragile]
\frametitle{Lists}
\begin{block}{a list using Haskell}
\begin{lstlisting}[style=haskell,basicstyle=\scriptsize\ttfamily,mathescape]
data List a = Nil | Cons a (List a)
\end{lstlisting}
\end{block}
\tiny{\emph{\textbf{never ever anything else} is enforced in haskell}}
\end{frame}

\begin{frame}[fragile]
\frametitle{Some examples of Lists}
\begin{block}{C\#}
\begin{lstlisting}[style=csharp,basicstyle=\scriptsize\ttfamily,mathescape]
new Cons<int>(12, new Nil<int>())
\end{lstlisting}
\end{block}
\begin{block}{Haskell}
\begin{lstlisting}[style=haskell,basicstyle=\scriptsize\ttfamily,mathescape]
Cons 12 Nil
\end{lstlisting}
\end{block}
\end{frame}

\begin{frame}[fragile]
\frametitle{Some examples of Lists}
\begin{block}{C\#}
\begin{lstlisting}[style=csharp,basicstyle=\tiny\ttfamily,mathescape]
new Cons<char>('a', new Cons<char>('b', new Cons<char>('c', new Nil<char>())))
\end{lstlisting}
\end{block}
\begin{block}{Haskell}
\begin{lstlisting}[style=haskell,basicstyle=\scriptsize\ttfamily,mathescape]
Cons 'a' (Cons 'b' (Cons 'c' Nil))
\end{lstlisting}
\end{block}
\end{frame}

\begin{frame}
\frametitle{Some nomenclature}
\begin{block}{Naming Schmaming}
\begin{itemize}
\item<1-> Sometimes you will see \lstinline{Nil} denoted \lstinline{[]}
\item<2-> and/or \lstinline{Cons} denoted \lstinline{:} in an infix position
\item<3-> like this \lstinline{1:(2:(3:[]))}
\item<4-> but this is the same data structure
\end{itemize}
\end{block}
\end{frame}

\begin{frame}
\frametitle{Lists}
\begin{center}
Ensure we all know what a list is
\end{center}
\end{frame}
